% Options for packages loaded elsewhere
\PassOptionsToPackage{unicode}{hyperref}
\PassOptionsToPackage{hyphens}{url}
%
\documentclass[
]{article}
\usepackage{amsmath,amssymb}
\usepackage{iftex}
\ifPDFTeX
  \usepackage[T1]{fontenc}
  \usepackage[utf8]{inputenc}
  \usepackage{textcomp} % provide euro and other symbols
\else % if luatex or xetex
  \usepackage{unicode-math} % this also loads fontspec
  \defaultfontfeatures{Scale=MatchLowercase}
  \defaultfontfeatures[\rmfamily]{Ligatures=TeX,Scale=1}
\fi
\usepackage{lmodern}
\ifPDFTeX\else
  % xetex/luatex font selection
\fi
% Use upquote if available, for straight quotes in verbatim environments
\IfFileExists{upquote.sty}{\usepackage{upquote}}{}
\IfFileExists{microtype.sty}{% use microtype if available
  \usepackage[]{microtype}
  \UseMicrotypeSet[protrusion]{basicmath} % disable protrusion for tt fonts
}{}
\makeatletter
\@ifundefined{KOMAClassName}{% if non-KOMA class
  \IfFileExists{parskip.sty}{%
    \usepackage{parskip}
  }{% else
    \setlength{\parindent}{0pt}
    \setlength{\parskip}{6pt plus 2pt minus 1pt}}
}{% if KOMA class
  \KOMAoptions{parskip=half}}
\makeatother
\usepackage{xcolor}
\usepackage[margin=1in]{geometry}
\usepackage{color}
\usepackage{fancyvrb}
\newcommand{\VerbBar}{|}
\newcommand{\VERB}{\Verb[commandchars=\\\{\}]}
\DefineVerbatimEnvironment{Highlighting}{Verbatim}{commandchars=\\\{\}}
% Add ',fontsize=\small' for more characters per line
\usepackage{framed}
\definecolor{shadecolor}{RGB}{248,248,248}
\newenvironment{Shaded}{\begin{snugshade}}{\end{snugshade}}
\newcommand{\AlertTok}[1]{\textcolor[rgb]{0.94,0.16,0.16}{#1}}
\newcommand{\AnnotationTok}[1]{\textcolor[rgb]{0.56,0.35,0.01}{\textbf{\textit{#1}}}}
\newcommand{\AttributeTok}[1]{\textcolor[rgb]{0.13,0.29,0.53}{#1}}
\newcommand{\BaseNTok}[1]{\textcolor[rgb]{0.00,0.00,0.81}{#1}}
\newcommand{\BuiltInTok}[1]{#1}
\newcommand{\CharTok}[1]{\textcolor[rgb]{0.31,0.60,0.02}{#1}}
\newcommand{\CommentTok}[1]{\textcolor[rgb]{0.56,0.35,0.01}{\textit{#1}}}
\newcommand{\CommentVarTok}[1]{\textcolor[rgb]{0.56,0.35,0.01}{\textbf{\textit{#1}}}}
\newcommand{\ConstantTok}[1]{\textcolor[rgb]{0.56,0.35,0.01}{#1}}
\newcommand{\ControlFlowTok}[1]{\textcolor[rgb]{0.13,0.29,0.53}{\textbf{#1}}}
\newcommand{\DataTypeTok}[1]{\textcolor[rgb]{0.13,0.29,0.53}{#1}}
\newcommand{\DecValTok}[1]{\textcolor[rgb]{0.00,0.00,0.81}{#1}}
\newcommand{\DocumentationTok}[1]{\textcolor[rgb]{0.56,0.35,0.01}{\textbf{\textit{#1}}}}
\newcommand{\ErrorTok}[1]{\textcolor[rgb]{0.64,0.00,0.00}{\textbf{#1}}}
\newcommand{\ExtensionTok}[1]{#1}
\newcommand{\FloatTok}[1]{\textcolor[rgb]{0.00,0.00,0.81}{#1}}
\newcommand{\FunctionTok}[1]{\textcolor[rgb]{0.13,0.29,0.53}{\textbf{#1}}}
\newcommand{\ImportTok}[1]{#1}
\newcommand{\InformationTok}[1]{\textcolor[rgb]{0.56,0.35,0.01}{\textbf{\textit{#1}}}}
\newcommand{\KeywordTok}[1]{\textcolor[rgb]{0.13,0.29,0.53}{\textbf{#1}}}
\newcommand{\NormalTok}[1]{#1}
\newcommand{\OperatorTok}[1]{\textcolor[rgb]{0.81,0.36,0.00}{\textbf{#1}}}
\newcommand{\OtherTok}[1]{\textcolor[rgb]{0.56,0.35,0.01}{#1}}
\newcommand{\PreprocessorTok}[1]{\textcolor[rgb]{0.56,0.35,0.01}{\textit{#1}}}
\newcommand{\RegionMarkerTok}[1]{#1}
\newcommand{\SpecialCharTok}[1]{\textcolor[rgb]{0.81,0.36,0.00}{\textbf{#1}}}
\newcommand{\SpecialStringTok}[1]{\textcolor[rgb]{0.31,0.60,0.02}{#1}}
\newcommand{\StringTok}[1]{\textcolor[rgb]{0.31,0.60,0.02}{#1}}
\newcommand{\VariableTok}[1]{\textcolor[rgb]{0.00,0.00,0.00}{#1}}
\newcommand{\VerbatimStringTok}[1]{\textcolor[rgb]{0.31,0.60,0.02}{#1}}
\newcommand{\WarningTok}[1]{\textcolor[rgb]{0.56,0.35,0.01}{\textbf{\textit{#1}}}}
\usepackage{graphicx}
\makeatletter
\newsavebox\pandoc@box
\newcommand*\pandocbounded[1]{% scales image to fit in text height/width
  \sbox\pandoc@box{#1}%
  \Gscale@div\@tempa{\textheight}{\dimexpr\ht\pandoc@box+\dp\pandoc@box\relax}%
  \Gscale@div\@tempb{\linewidth}{\wd\pandoc@box}%
  \ifdim\@tempb\p@<\@tempa\p@\let\@tempa\@tempb\fi% select the smaller of both
  \ifdim\@tempa\p@<\p@\scalebox{\@tempa}{\usebox\pandoc@box}%
  \else\usebox{\pandoc@box}%
  \fi%
}
% Set default figure placement to htbp
\def\fps@figure{htbp}
\makeatother
\setlength{\emergencystretch}{3em} % prevent overfull lines
\providecommand{\tightlist}{%
  \setlength{\itemsep}{0pt}\setlength{\parskip}{0pt}}
\setcounter{secnumdepth}{-\maxdimen} % remove section numbering
\usepackage{bookmark}
\IfFileExists{xurl.sty}{\usepackage{xurl}}{} % add URL line breaks if available
\urlstyle{same}
\hypersetup{
  pdftitle={Tarea\_1},
  pdfauthor={Juan\_Urbina},
  hidelinks,
  pdfcreator={LaTeX via pandoc}}

\title{Tarea\_1}
\author{Juan\_Urbina}
\date{2025-06-06}

\begin{document}
\maketitle

\section{Parte 1}\label{parte-1}

Escriba el código de R necesario para resolver los siguientes incisos
trate de usar la menor cantidad de líneas posible.

\begin{enumerate}
\def\labelenumi{\arabic{enumi}.}
\tightlist
\item
  ¿Cómo generaría un arreglo aleatorio de 250 elementos con los colores
  primarios?
\end{enumerate}

\begin{Shaded}
\begin{Highlighting}[]
\NormalTok{ejer\_1 }\OtherTok{\textless{}{-}} \FunctionTok{sample}\NormalTok{(}\FunctionTok{c}\NormalTok{(}\StringTok{"rojo"}\NormalTok{, }\StringTok{"azul"}\NormalTok{, }\StringTok{"amarillo"}\NormalTok{), }\DecValTok{250}\NormalTok{, }\AttributeTok{replace =} \ConstantTok{TRUE}\NormalTok{)}
\NormalTok{ejer\_1}
\end{Highlighting}
\end{Shaded}

\begin{verbatim}
##   [1] "rojo"     "amarillo" "rojo"     "rojo"     "rojo"     "rojo"    
##   [7] "azul"     "amarillo" "rojo"     "amarillo" "azul"     "amarillo"
##  [13] "azul"     "rojo"     "rojo"     "rojo"     "amarillo" "amarillo"
##  [19] "azul"     "azul"     "amarillo" "rojo"     "rojo"     "azul"    
##  [25] "rojo"     "azul"     "amarillo" "rojo"     "azul"     "amarillo"
##  [31] "rojo"     "azul"     "azul"     "rojo"     "amarillo" "azul"    
##  [37] "rojo"     "amarillo" "azul"     "azul"     "rojo"     "amarillo"
##  [43] "azul"     "amarillo" "rojo"     "azul"     "azul"     "amarillo"
##  [49] "azul"     "amarillo" "azul"     "azul"     "rojo"     "azul"    
##  [55] "azul"     "azul"     "azul"     "amarillo" "azul"     "azul"    
##  [61] "rojo"     "azul"     "rojo"     "rojo"     "amarillo" "amarillo"
##  [67] "azul"     "azul"     "azul"     "rojo"     "amarillo" "azul"    
##  [73] "azul"     "rojo"     "azul"     "amarillo" "rojo"     "amarillo"
##  [79] "azul"     "amarillo" "amarillo" "azul"     "rojo"     "rojo"    
##  [85] "rojo"     "azul"     "amarillo" "amarillo" "amarillo" "rojo"    
##  [91] "azul"     "rojo"     "amarillo" "amarillo" "amarillo" "rojo"    
##  [97] "azul"     "amarillo" "azul"     "rojo"     "amarillo" "rojo"    
## [103] "amarillo" "azul"     "amarillo" "rojo"     "rojo"     "rojo"    
## [109] "azul"     "azul"     "azul"     "rojo"     "amarillo" "azul"    
## [115] "amarillo" "rojo"     "rojo"     "rojo"     "azul"     "amarillo"
## [121] "azul"     "rojo"     "amarillo" "rojo"     "amarillo" "amarillo"
## [127] "amarillo" "azul"     "rojo"     "azul"     "azul"     "azul"    
## [133] "amarillo" "amarillo" "azul"     "rojo"     "amarillo" "azul"    
## [139] "azul"     "amarillo" "rojo"     "azul"     "amarillo" "azul"    
## [145] "amarillo" "rojo"     "azul"     "rojo"     "rojo"     "azul"    
## [151] "rojo"     "rojo"     "azul"     "azul"     "rojo"     "azul"    
## [157] "rojo"     "rojo"     "rojo"     "amarillo" "amarillo" "rojo"    
## [163] "azul"     "amarillo" "azul"     "azul"     "azul"     "azul"    
## [169] "rojo"     "azul"     "amarillo" "rojo"     "rojo"     "rojo"    
## [175] "rojo"     "azul"     "rojo"     "azul"     "azul"     "amarillo"
## [181] "rojo"     "amarillo" "rojo"     "amarillo" "amarillo" "amarillo"
## [187] "amarillo" "azul"     "azul"     "rojo"     "amarillo" "azul"    
## [193] "amarillo" "rojo"     "amarillo" "azul"     "azul"     "azul"    
## [199] "rojo"     "rojo"     "azul"     "azul"     "amarillo" "rojo"    
## [205] "rojo"     "azul"     "amarillo" "rojo"     "azul"     "azul"    
## [211] "azul"     "azul"     "azul"     "rojo"     "amarillo" "amarillo"
## [217] "amarillo" "rojo"     "azul"     "azul"     "azul"     "amarillo"
## [223] "amarillo" "amarillo" "amarillo" "azul"     "amarillo" "azul"    
## [229] "amarillo" "azul"     "amarillo" "amarillo" "amarillo" "amarillo"
## [235] "rojo"     "amarillo" "azul"     "azul"     "rojo"     "amarillo"
## [241] "azul"     "azul"     "azul"     "amarillo" "amarillo" "azul"    
## [247] "amarillo" "azul"     "azul"     "amarillo"
\end{verbatim}

\begin{enumerate}
\def\labelenumi{\arabic{enumi}.}
\setcounter{enumi}{1}
\tightlist
\item
  Dado un arreglo de 120000 elementos de números enteros entre 40 y 70,
  ¿Cómo haría para obtener la desviación estándar de aquellos números
  que son mayores a 55 y menores a 64?
\end{enumerate}

\begin{Shaded}
\begin{Highlighting}[]
\NormalTok{ejer\_2 }\OtherTok{\textless{}{-}} \FunctionTok{sd}\NormalTok{(}\FunctionTok{sample}\NormalTok{(}\DecValTok{40}\SpecialCharTok{:}\DecValTok{70}\NormalTok{, }\DecValTok{120000}\NormalTok{, }\ConstantTok{TRUE}\NormalTok{)[}\FunctionTok{sample}\NormalTok{(}\DecValTok{40}\SpecialCharTok{:}\DecValTok{70}\NormalTok{, }\DecValTok{120000}\NormalTok{, }\ConstantTok{TRUE}\NormalTok{) }\SpecialCharTok{\textgreater{}} \DecValTok{55} \SpecialCharTok{\&} \FunctionTok{sample}\NormalTok{(}\DecValTok{40}\SpecialCharTok{:}\DecValTok{70}\NormalTok{, }\DecValTok{120000}\NormalTok{, }\ConstantTok{TRUE}\NormalTok{) }\SpecialCharTok{\textless{}} \DecValTok{64}\NormalTok{])}
\NormalTok{ejer\_2}
\end{Highlighting}
\end{Shaded}

\begin{verbatim}
## [1] 8.966839
\end{verbatim}

\begin{enumerate}
\def\labelenumi{\arabic{enumi}.}
\setcounter{enumi}{2}
\tightlist
\item
  Sabemos que para sumar vectores estos deben tener la misma longitud.
  Sin embargo, R trabaja de manera distinta. Defina los vectores x = (1,
  2, 3, 4, 5, 6), y = (7, 8), z = (9, 10, 11, 12).
\end{enumerate}

\begin{enumerate}
\def\labelenumi{\alph{enumi}.}
\tightlist
\item
  Calcula:

  \begin{enumerate}
  \def\labelenumii{\roman{enumii}.}
  \tightlist
  \item
    x + x
  \item
    x + y.
  \item
    Responda ¿Qué ha hecho R?
  \end{enumerate}
\end{enumerate}

\begin{Shaded}
\begin{Highlighting}[]
\NormalTok{x }\OtherTok{\textless{}{-}} \DecValTok{1}\SpecialCharTok{:}\DecValTok{6}
\NormalTok{y }\OtherTok{\textless{}{-}} \DecValTok{7}\SpecialCharTok{:}\DecValTok{8}\NormalTok{; z }\OtherTok{\textless{}{-}} \DecValTok{9}\SpecialCharTok{:}\DecValTok{12}

\CommentTok{\# Parte (i)}
\NormalTok{x }\SpecialCharTok{+}\NormalTok{ x}
\end{Highlighting}
\end{Shaded}

\begin{verbatim}
## [1]  2  4  6  8 10 12
\end{verbatim}

\begin{Shaded}
\begin{Highlighting}[]
\CommentTok{\# Parte (ii)}
\NormalTok{x }\SpecialCharTok{+}\NormalTok{ y}
\end{Highlighting}
\end{Shaded}

\begin{verbatim}
## [1]  8 10 10 12 12 14
\end{verbatim}

\begin{Shaded}
\begin{Highlighting}[]
\CommentTok{\# Parte(iii)}
\DocumentationTok{\#\# R recicla los valores de \textquotesingle{}y\textquotesingle{} para completar la longitud de \textquotesingle{}x\textquotesingle{}}
\end{Highlighting}
\end{Shaded}

\begin{enumerate}
\def\labelenumi{\arabic{enumi}.}
\setcounter{enumi}{3}
\tightlist
\item
  Si b \textless- list(a=1:10, c=``Hola'', d=``XX''), escriba una
  expresión en R que devuelva todos los elementos de la lista excepto
  los elementos 2,4,6,7,8 del vector a.
\end{enumerate}

\begin{Shaded}
\begin{Highlighting}[]
\NormalTok{b }\OtherTok{\textless{}{-}} \FunctionTok{list}\NormalTok{(}\AttributeTok{a=}\DecValTok{1}\SpecialCharTok{:}\DecValTok{10}\NormalTok{, }\AttributeTok{c=}\StringTok{"Hola"}\NormalTok{, }\AttributeTok{d=}\StringTok{"XX"}\NormalTok{)}

\NormalTok{b}\SpecialCharTok{$}\NormalTok{a[}\SpecialCharTok{!}\DecValTok{1}\SpecialCharTok{:}\DecValTok{10} \SpecialCharTok{\%in\%} \FunctionTok{c}\NormalTok{(}\DecValTok{2}\NormalTok{,}\DecValTok{4}\NormalTok{,}\DecValTok{6}\NormalTok{,}\DecValTok{7}\NormalTok{,}\DecValTok{8}\NormalTok{)]}
\end{Highlighting}
\end{Shaded}

\begin{verbatim}
## [1]  1  3  5  9 10
\end{verbatim}

\begin{enumerate}
\def\labelenumi{\arabic{enumi}.}
\setcounter{enumi}{4}
\tightlist
\item
  Suponga que se almacena en una variable un arreglo con todos los
  números de carnet de los estudiantes de la Escuela de Ciencias y
  Sistemas en forma de string, los carnets tienen al inicio cuatro
  caracteres que identifican el año en el que la persona entró a la
  universidad, por ejemplo: 201900452 hacer referencia a una persona que
  ingreso en el año 2019, 200600987 hace referencia a una persona que
  ingreso que el año 2006.
\end{enumerate}

\begin{enumerate}
\def\labelenumi{\alph{enumi}.}
\tightlist
\item
  ¿Cómo haría para contar la cantidad de alumnos por año?
\end{enumerate}

\begin{Shaded}
\begin{Highlighting}[]
\NormalTok{carnets }\OtherTok{\textless{}{-}} \FunctionTok{c}\NormalTok{(}\StringTok{"201900452"}\NormalTok{, }\StringTok{"201800321"}\NormalTok{, }\StringTok{"201900987"}\NormalTok{, }\StringTok{"200600123"}\NormalTok{, }\StringTok{"200600987"}\NormalTok{, }
             \StringTok{"201700111"}\NormalTok{, }\StringTok{"201800222"}\NormalTok{, }\StringTok{"201900333"}\NormalTok{, }\StringTok{"202000555"}\NormalTok{, }\StringTok{"202000666"}\NormalTok{)}

\FunctionTok{table}\NormalTok{(}\FunctionTok{substr}\NormalTok{(carnets, }\DecValTok{1}\NormalTok{, }\DecValTok{4}\NormalTok{))}
\end{Highlighting}
\end{Shaded}

\begin{verbatim}
## 
## 2006 2017 2018 2019 2020 
##    2    1    2    3    2
\end{verbatim}

\begin{enumerate}
\def\labelenumi{\arabic{enumi}.}
\setcounter{enumi}{5}
\tightlist
\item
  Dada la variable string \textless- ``Hola Mundo'', escriba una
  instrucción en R que devuelva la siguiente salida: {[}{[}1{]}{]}
  {[}1{]} ``Hola'' {[}{[}2{]}{]} {[}1{]} ``Mundo''.
\end{enumerate}

\begin{Shaded}
\begin{Highlighting}[]
\NormalTok{string }\OtherTok{\textless{}{-}} \FunctionTok{strsplit}\NormalTok{(}\StringTok{"Hola Mundo"}\NormalTok{, }\StringTok{" "}\NormalTok{)}

\NormalTok{string}
\end{Highlighting}
\end{Shaded}

\begin{verbatim}
## [[1]]
## [1] "Hola"  "Mundo"
\end{verbatim}

\begin{Shaded}
\begin{Highlighting}[]
\NormalTok{string[[}\DecValTok{1}\NormalTok{]][}\DecValTok{1}\NormalTok{]}
\end{Highlighting}
\end{Shaded}

\begin{verbatim}
## [1] "Hola"
\end{verbatim}

\begin{Shaded}
\begin{Highlighting}[]
\NormalTok{string[[}\DecValTok{1}\NormalTok{]][}\DecValTok{2}\NormalTok{]}
\end{Highlighting}
\end{Shaded}

\begin{verbatim}
## [1] "Mundo"
\end{verbatim}

\begin{enumerate}
\def\labelenumi{\arabic{enumi}.}
\setcounter{enumi}{6}
\tightlist
\item
  Utilizando el dataset flights de la librería nycflights23 (recordando:
  dataset\textless- flights), realizar lo siguiente:
\end{enumerate}

\begin{enumerate}
\def\labelenumi{\alph{enumi}.}
\tightlist
\item
  Mostrar una gráfica de barras de manera decreciente con la cantidad de
  vuelo por aerolínea (columna ``carrier'')
\item
  Mostrar una gráfica de líneas de cantidad de vuelos por mes (columna
  ``month'').
\end{enumerate}

\begin{Shaded}
\begin{Highlighting}[]
\FunctionTok{library}\NormalTok{(nycflights13)}
\FunctionTok{library}\NormalTok{(dplyr)}
\end{Highlighting}
\end{Shaded}

\begin{verbatim}
## 
## Adjuntando el paquete: 'dplyr'
\end{verbatim}

\begin{verbatim}
## The following objects are masked from 'package:stats':
## 
##     filter, lag
\end{verbatim}

\begin{verbatim}
## The following objects are masked from 'package:base':
## 
##     intersect, setdiff, setequal, union
\end{verbatim}

\begin{Shaded}
\begin{Highlighting}[]
\FunctionTok{library}\NormalTok{(ggplot2)}

\CommentTok{\# 7.a}
\NormalTok{flights }\SpecialCharTok{\%\textgreater{}\%} \FunctionTok{count}\NormalTok{(carrier, }\AttributeTok{sort=}\ConstantTok{TRUE}\NormalTok{) }\SpecialCharTok{\%\textgreater{}\%}
  \FunctionTok{ggplot}\NormalTok{(}\FunctionTok{aes}\NormalTok{(}\AttributeTok{x=}\FunctionTok{reorder}\NormalTok{(carrier, }\SpecialCharTok{{-}}\NormalTok{n), }\AttributeTok{y=}\NormalTok{n)) }\SpecialCharTok{+} \FunctionTok{geom\_bar}\NormalTok{(}\AttributeTok{stat=}\StringTok{"identity"}\NormalTok{)}
\end{Highlighting}
\end{Shaded}

\pandocbounded{\includegraphics[keepaspectratio]{\%5BMyS1\%5D_\%5B201906051\%5D_files/figure-latex/unnamed-chunk-7-1.pdf}}

\begin{Shaded}
\begin{Highlighting}[]
\CommentTok{\# 7.b}
\NormalTok{flights }\SpecialCharTok{\%\textgreater{}\%} \FunctionTok{count}\NormalTok{(month) }\SpecialCharTok{\%\textgreater{}\%}
  \FunctionTok{ggplot}\NormalTok{(}\FunctionTok{aes}\NormalTok{(}\AttributeTok{x=}\NormalTok{month, }\AttributeTok{y=}\NormalTok{n)) }\SpecialCharTok{+} \FunctionTok{geom\_line}\NormalTok{()}
\end{Highlighting}
\end{Shaded}

\pandocbounded{\includegraphics[keepaspectratio]{\%5BMyS1\%5D_\%5B201906051\%5D_files/figure-latex/unnamed-chunk-8-1.pdf}}

\section{Parte 2}\label{parte-2}

\subsection{Ejercicio 1: Lanzamiento de dos
dados}\label{ejercicio-1-lanzamiento-de-dos-dados}

Lanza dos dados justos. Define el espacio muestral y calcula las
siguientes probabilidades: A. Que la suma sea 7. B. Que ambos dados
muestren el mismo número. C. Que la suma sea 7 o los dados muestren el
mismo número.

\begin{Shaded}
\begin{Highlighting}[]
\NormalTok{muestral }\OtherTok{\textless{}{-}} \FunctionTok{expand.grid}\NormalTok{(}\AttributeTok{d1 =} \DecValTok{1}\SpecialCharTok{:}\DecValTok{6}\NormalTok{, }\AttributeTok{d2 =} \DecValTok{1}\SpecialCharTok{:}\DecValTok{6}\NormalTok{)}
\NormalTok{A }\OtherTok{\textless{}{-}} \FunctionTok{with}\NormalTok{(muestral, d1 }\SpecialCharTok{+}\NormalTok{ d2 }\SpecialCharTok{==} \DecValTok{7}\NormalTok{)}
\NormalTok{B }\OtherTok{\textless{}{-}} \FunctionTok{with}\NormalTok{(muestral, d1 }\SpecialCharTok{==}\NormalTok{ d2)}
\NormalTok{C }\OtherTok{\textless{}{-}}\NormalTok{ A }\SpecialCharTok{|}\NormalTok{ B}
\FunctionTok{c}\NormalTok{(}\AttributeTok{P\_A =} \FunctionTok{mean}\NormalTok{(A), }\AttributeTok{P\_B =} \FunctionTok{mean}\NormalTok{(B), }\AttributeTok{P\_C =} \FunctionTok{mean}\NormalTok{(C))}
\end{Highlighting}
\end{Shaded}

\begin{verbatim}
##       P_A       P_B       P_C 
## 0.1666667 0.1666667 0.3333333
\end{verbatim}

\subsection{Ejercicio 2: Lanzamiento de tres
monedas}\label{ejercicio-2-lanzamiento-de-tres-monedas}

Lanza tres monedas justas. Simula este experimento 1000 veces y calcula:
A. La probabilidad de que salgan exactamente dos caras. B. La
probabilidad de que al menos una sea sello. C. La probabilidad de que
todas sean iguales.

\begin{Shaded}
\begin{Highlighting}[]
\FunctionTok{set.seed}\NormalTok{(}\DecValTok{1}\NormalTok{)}

\NormalTok{monedas }\OtherTok{\textless{}{-}} \FunctionTok{replicate}\NormalTok{(}\DecValTok{1000}\NormalTok{, }\FunctionTok{sample}\NormalTok{(}\FunctionTok{c}\NormalTok{(}\StringTok{"c"}\NormalTok{, }\StringTok{"s"}\NormalTok{), }\DecValTok{3}\NormalTok{, }\AttributeTok{replace=}\ConstantTok{TRUE}\NormalTok{))}

\NormalTok{caras }\OtherTok{\textless{}{-}} \FunctionTok{apply}\NormalTok{(monedas, }\DecValTok{2}\NormalTok{, }\ControlFlowTok{function}\NormalTok{(x) }\FunctionTok{sum}\NormalTok{(x }\SpecialCharTok{==} \StringTok{"c"}\NormalTok{))}

\FunctionTok{c}\NormalTok{(}
  \AttributeTok{P\_dos\_caras =} \FunctionTok{mean}\NormalTok{(caras }\SpecialCharTok{==} \DecValTok{2}\NormalTok{),}
  \AttributeTok{P\_al\_menos\_una\_sello =} \FunctionTok{mean}\NormalTok{(}\FunctionTok{apply}\NormalTok{(monedas, }\DecValTok{2}\NormalTok{, }\ControlFlowTok{function}\NormalTok{(x) }\StringTok{"s"} \SpecialCharTok{\%in\%}\NormalTok{ x)),}
  \AttributeTok{P\_todas\_iguales =} \FunctionTok{mean}\NormalTok{(}\FunctionTok{apply}\NormalTok{(monedas, }\DecValTok{2}\NormalTok{, }\ControlFlowTok{function}\NormalTok{(x) }\FunctionTok{length}\NormalTok{(}\FunctionTok{unique}\NormalTok{(x)) }\SpecialCharTok{==} \DecValTok{1}\NormalTok{))}
\NormalTok{)}
\end{Highlighting}
\end{Shaded}

\begin{verbatim}
##          P_dos_caras P_al_menos_una_sello      P_todas_iguales 
##                0.373                0.878                0.242
\end{verbatim}

\subsection{Ejercicio 3: Se extrae una carta de una baraja estándar (52
cartas).
Calcular:}\label{ejercicio-3-se-extrae-una-carta-de-una-baraja-estuxe1ndar-52-cartas.-calcular}

A. A: la carta es de corazón. B. B: la carta es una figura (J, Q o K).
C. ¿Son independientes los eventos A y B?

\begin{Shaded}
\begin{Highlighting}[]
\NormalTok{baraja }\OtherTok{\textless{}{-}} \FunctionTok{expand.grid}\NormalTok{(}\AttributeTok{valor=}\FunctionTok{c}\NormalTok{(}\DecValTok{1}\SpecialCharTok{:}\DecValTok{10}\NormalTok{, }\StringTok{"J"}\NormalTok{, }\StringTok{"Q"}\NormalTok{, }\StringTok{"K"}\NormalTok{), }\AttributeTok{palo=}\FunctionTok{c}\NormalTok{(}\StringTok{"Corazon"}\NormalTok{, }\StringTok{"Diamante"}\NormalTok{, }\StringTok{"Trebol"}\NormalTok{, }\StringTok{"Espada"}\NormalTok{))}

\NormalTok{A }\OtherTok{\textless{}{-}}\NormalTok{ baraja}\SpecialCharTok{$}\NormalTok{palo }\SpecialCharTok{==} \StringTok{"Corazon"}
\NormalTok{B }\OtherTok{\textless{}{-}}\NormalTok{ baraja}\SpecialCharTok{$}\NormalTok{valor }\SpecialCharTok{\%in\%} \FunctionTok{c}\NormalTok{(}\StringTok{"J"}\NormalTok{, }\StringTok{"Q"}\NormalTok{, }\StringTok{"K"}\NormalTok{)}

\NormalTok{independientes }\OtherTok{\textless{}{-}} \FunctionTok{mean}\NormalTok{(A }\SpecialCharTok{\&}\NormalTok{ B) }\SpecialCharTok{==} \FunctionTok{mean}\NormalTok{(A) }\SpecialCharTok{*} \FunctionTok{mean}\NormalTok{(B)}

\FunctionTok{list}\NormalTok{(}\AttributeTok{P\_A =} \FunctionTok{mean}\NormalTok{(A), }\AttributeTok{P\_B =} \FunctionTok{mean}\NormalTok{(B), }\AttributeTok{Independientes =}\NormalTok{ independientes)}
\end{Highlighting}
\end{Shaded}

\begin{verbatim}
## $P_A
## [1] 0.25
## 
## $P_B
## [1] 0.2307692
## 
## $Independientes
## [1] TRUE
\end{verbatim}

\end{document}
